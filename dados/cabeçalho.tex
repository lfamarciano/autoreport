\documentclass{article}
\usepackage{graphicx} % Required for inserting images
\usepackage{hyperref}
\usepackage[utf8]{inputenc}
\usepackage{amsfonts}
\usepackage{amsmath}
\usepackage{amssymb}
\usepackage{xfrac}
\usepackage{cancel}

\title{Lista 1 Cálculo}
\author{Luís Felipe de Abreu Marciano}
\date{\today}

\usepackage[brazil]{babel}

\begin{document}

\maketitle

\newpage

\tableofcontents

\newpage

\section{ }
\subsection*{a)}
$a_1 = \frac{2}{2+1} = \frac{2}{3}\\
a_2 = \frac{2^2}{4+1} = \frac{4}{5}\\
a_3 = \frac{8}{7}\\
a_4 = \frac{16}{9}\\
a_5 = \frac{32}{11}$

\subsection*{b)}
$a_1 = 0\\
a_2 = \frac{3}{5}\\
a_3 = \frac{4}{5}\\
a_4 = \frac{15}{17}\\
a_5 = \frac{12}{13}$

\subsection*{c)}
$a_1 = 1\\
a_2 = \frac{4}{5}\\
a_3 = \frac{3}{5}\\
a_4 = \frac{8}{17}\\
a_5 = \frac{5}{13}$

\subsection*{d)}
$a_1 = 6\\
a_2 = 6\\
a_3 = 3\\
a_4 = 1\\
a_5 = \frac{1}{4}$

\section{}
\subsection*{a)}
$a_n = \frac{1}{2n}\\$

\subsection*{b)}
$a_n = -16 \cdot \left(-\frac{1}{4}\right)^n\\$

\subsection*{c)}
$a_n = i^{n+3} \cdot \frac{(-1)^{n + 1}+ 1}{2}\\$

\section{}

\subsection*{a)}
$\lim a_n = \lim \frac{3+5n^2}{n+n^2} = \lim (\frac{3}{n+n^2} + \frac{5n^2 + 5 - 5}{n(n+1)}) = \lim \frac{5\cancel{(n+1)}}{n\cancel{(n+1)}} = \lim \frac{5}{n} = 5 \\$
Portanto, $a_n$ é convergente.

\subsection*{b)}
$\lim a_n = \lim \frac{3+5n^2}{1+n^2} = \lim \frac{5(n^2 + 1) - 2}{n^2 + 1} = \lim (-\frac{2}{n^2+1} + 5\frac{\cancel{n^2+1}}{\cancel{n^2+1}}) = 5\\$
Portanto, $a_n$ é convergente.

\subsection*{c)}
$0 \leq cos^2{n} \leq 1 \\
\lim a_n = \lim \frac{cos^2{n}}{n} = 0\\$ 
Portanto, $a_n$ é convergente.

\subsection*{d)}
$a_n = 
\begin{cases}
    \frac{n^3}{n^3 + 2n +1} \text{ se $n = 2k$}\\
    -\frac{n^3}{n^3 + 2n +1} \text{ se $n = 2k -1$}
\end{cases}\\
\lim a_{2k} = \lim \frac{n^3}{n^3 + 2n + 1} = \lim \frac{1}{1 + \frac{2}{n^2} + \frac{1}{n^3}} = 1\\
\lim a_{2k-1} = \lim -\frac{n^3}{n^3 + 2n + 1} = \lim -\frac{1}{1 + \frac{2}{n^2} + \frac{1}{n^3}} = -1\\$
Portanto, como $\lim a_{2k} \neq \lim a_{2k-1} $, $a_n$ é divergente.

\subsection*{e)}
$a_n = 
\begin{cases}
    \frac{n}{n^2+1} \text{ se $n = 2k$}\\
    -\frac{n}{n^2+1} \text{ se $n = 2k-1$}
\end{cases}\\
\lim a_{2k} = \lim \frac{n}{n^2+1}=\lim \frac{1}{n + \frac{1}{n}} = \lim \frac{1}{n} = 0 = \lim a_{2k-1}$\\
Portanto, $a_n$ é convergente.

\section{}
Essa sequência é contante, $a_n = c = \lim a_n$

\section{}
$|a_n-a| < \frac{1}{2^n}\\
\frac{1}{2^n} < \frac{1}{1000}\\
2^n > 1000\\
n \geq 10$ \hspace{30pt} \{$n \in \mathbb{R}/ n \geq 10$\}

\section{}
Em qualquer caso de $a_n = b_n$\\
Exemplo: $a_n = b_n = n^2 + 2n$

\section{}
\subsection*{i)} A proposição é verdadeira

\subsection*{ii)} b)

\section{}
\subsection*{a)} Sim, pois apesar de $x_n$ ser limitada, $0 \leq x_n \leq 2$, ela não apresenta limite.
\subsection*{b)} Sim, pois apesar de $x_n$ ser limitada, $-1 < x_n \leq 2$, ela não apresenta limite.
\subsection*{c)} Não, pois $x_n$ não é limitada.
\subsection*{d)} Não, pois $x_n$ não é limitada.

\section{}
A afirmação b) é a correta.

\section{}
A afirmação c) é a correta.

\section{}
$a_n = 2n + 6\\
b_n = -2n\\
\lim (a_n + b_n) = \lim (2n + 6 - 2n) = 6$

\section{}
\subsection*{a)}
$a_n = n\\
b_n = n^2\\
\lim \frac{a_n}{b_n} = \lim \frac{n}{n^2} = \lim \frac{1}{n} = 0$

\subsection*{b)}
$a_n = n - 5,8n^2\\
b_n = n^2\\
\lim \frac{a_n}{b_n} = \lim \frac{n-5,8n^2}{n^2} = \lim \left(\frac{1}{n} - 5,8\right) = 5,8$

\subsection*{c)}
$a_n = -n^2\\
b_n = n\\
\lim \frac{a_n}{b_n} = \lim \frac{-n^2}{n} = \lim -n = -\infty$


\section{}
\subsection*{a)}
$u_n$ é convergente.

$\lim u_n = \lim \frac{n^2+8}{n^2+100} = \lim \frac{1 + \frac{8}{n^2}}{1+\frac{100}{n^2}} = 1$

\subsection*{b)}
$u_n$ é divergente.

\subsection*{c)}
$u_n$ é divergente.

\subsection*{d)}
$u_n$ é divergente.

\subsection*{e)}
$u_n$ é divergente.


\section{}
$a_n = (-1)^n + 1$\hspace{20pt} ($0 \leq a_n \leq 2$)$ \\
b_n = -(-1)^n $\hspace{30pt} ($ -1 \leq b_n \leq 1$)$ \\
\lim (a_n + b_n) = \lim (\cancel{(-1)^n} + 1 - \cancel{(-1)^n)} = 1$

\end{document}


